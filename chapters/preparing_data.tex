\chapter{Preparing data} \label{prep}

\section{Choosing your dataset}

There is a number of different topographic datasets and you will need to decide which is most appropriate for your project. LIDAR is preferable for detailed sudies with resolution of approx 10m. However to run the inversion, you will be looking to include large river drainage networks to stabalise the inversion. In this case 30 m (1 arc second) or 90 m (3 arc second) elevation datasets will be computationally less intensive. 

Computational power is one of the key considerations when considering the areal extent of your project. [Include section on why this needs to be considered carefully]

The following website has a nice compilation of some of the global LIDAR data available to use: \url{https://arheologijaslovenija.blogspot.com/p/blog-page_81.html?spref=tw}. For the area of interest in this particular project – the northern Arabian Platform – no LIDAR data was available. 

\subsection{SRTM data: Hatay}

NASA Shuttle Radar Topography Mission (SRTM) Global 1 arc second V003 granules were downloaded from \url{https://search.earthdata.nasa.gov} [on 26.01.19] (174 granules in total). The main objective was to ensure that the full length of the Euphrates and Tigris river originating in northeastern Turkey, draining via Syria and Iraq into the Persian Gulf. The aim is to test the hypothesis whether it is necessary for these large drainage networks to stabalise the inversion. 

However first to test the fidelity of drainage networks extracted using the suite of GRASS hydrological tools, this was first trialled on a small area of the Hatay Graben using the following 6 granules downloaded from the Earthdata NASA website:

\textit{N35E035.hgt
N35E036.hgt
N36E035.hgt
N36E036.hgt
N37E035.hgt
N37E036.hgt}


\subsection{Downloading grids from Nasa Earthdata using Linux}

The Earthdata website provides a ready-made script to download your chosen granules directly onto your computer. When working with large datasets, I would highly recommend option for this option as it will be hugely time efficient. Opt for direct download and click on ``Download Access Script``. This will take you to a page containing instructions on how to use the download script. Click the ``Download Script File`` button to download the script. From your download folder, open the script using your preferred text editor (e.g. Visual Studio Code, VIM, gedit):

\begin{lstlisting}[language=bash]
cd Downloads
code download.sh
\end{lstlisting}

To run the code, I need to make a small change ${\#}$!/bin/sh to ${\#}$!/bin/bash at the top of the script. Whether you need to do this or not will depend on your shell. In Ubuntu, \textit{bin/sh} no longer links to bash and instead points to another shell called dash. Save file and then execute and run the script from the command line in your Download folder by:

\begin{lstlisting}[language=bash]
chmod 777 download.sh 
./download.sh 
\end{lstlisting}

Type in your earthdata username and password when prompted. Be aware that nothing will show up when you type your password. The grids will now be automatically downloaded to our Downloads folder. Unzip files to chosen destination folder. If using a wild card to pattern match, make sure it is all within quote marks.

\begin{lstlisting}[language=bash]
unzip "*.hgt" destination-folder
\end{lstlisting}


\section{Merging rasters}

By far, the simplest and most time efficient way is to use gdal${\_}$merge.py instead of ArcGIS's \textit{``Mosaic to New Raster"} tool to mosaic each of the SRTM granules into one grid. For example, 176 1 arc second resolution SRTM grids covering the northern Arabian Peninsula was stiched together in less than a minute using:

\begin{lstlisting}[language=bash]
ls *.hgt > DEMs.txt
gdal_merge.py -n -32768 -a_nodata -32768 -o merged_dem.tif --optfile DEMs.txt
\end{lstlisting}

To find out information about your raster set, \textit{gdalinfo} is a very useful tool. This tool is used primarily to find out information on the raster’s coordinate system, resolution (i.e. pixel size) and extent of the raster through corner coordinates. No data needs to be specified if using RichDem to process the data. In order to mantain the NoData cells on the merged dem, it is necessary to specify a no data value. The -n flag ignores any pixels in the files being merged with this pixel value while -a_nodata assigns a specified nodata value to the output. You can check no data values in the original SRTM grids. 

\section{Projecting data}

Before importing your raster layer into GRASS, use gdalwarp to project the raster data into your preferred coordinate system. SRTM raster datasets use a geographic coordinate system based on a spherical surface. This can be problematic when measuring distances which uses angular units which depends on where you are on the earth’s surface. Extracting rivers relies on accurately knowing river distances measured in length, so use an equal areas projected coordinate. For small study areas, Universal Transvere Mercator (UTM) system is widely used as a projection system in metres. For continent-wide analyses, Albers Equal Areas or equivalent would be a good choice. Note that you will also need to consider the approproate ellipsoid. 

Projecting dem into UTM 37N (ESPG:32637) which is suitable for use with small narrow regions. The western end of Hatay Graben falls under UTM 36N, however given that boundaries of UTM Zone can extend ${\pm}$5 km, distortion will be small along the coastal region of Hatay. A simple way to understand the areal distortion as you move away from the UTM zone boundary is to use GeoConvert tool to determine the meridian convergence and scale in the correct UTM zone 36:

\begin{lstlisting}[language=bash]
echo "37N 229470.56706717 3983261.83838957" | GeoConvert -u -z 36 -c >> 1.76284 1.0005024
echo "37N 213040.77439037 3934458.42582466" | GeoConvert -u -z 36 -c >> 1.64826 1.0004153
\end{lstlisting}

For example at the mouth of riv 2, the area distortion 1.00100505240576 (1.0005024**2), while distortion at the mouth of riv 614 which represents the furthest coastal point from UTM Zone37 increases to 1.00083077247409. Distortion is relatively small so within acceptable bounds for this initial analysis. 

For the Hatay project, the merged dem was projected into Lambert Conformal Conic projection (similar to Albers Conic Equal Areas projection) but portrays shape more accurately than areas. Often used in the mid-latitudes. Reference latitudes were calculate for 1/6 and 5/6 respectively of the region of interest: S:29 and N:38 while the Deir ez Zor datum and Ellipsoid Clarke 1880 locally used in the Sryian which best represents the area of interest.

\begin{lstlisting}[language=bash]
gdalwarp -t_srs '+proj=lcc +lat_1=17.0 +lat_2=33.0 \
	+lat_0=25.08951 +lon_0=48.0 +ellps=intl +units=m \
	+no_defs' merged_dem.tif dem.tif
\end{lstlisting}




\section{Filling Voids}

The primary objective of the NASA MeaSUREs project (Making Earth System Data Records for Use in Research Environments) Program was to remove voids (no data holes) in the NASA SRTM DEM. In theory data should be seamless but there is a tool in GRASS to check and fill no data voids in the dem. In the meantime, you can use the flag in gdal \textit{dstnodata -9999} to fix any issues although this isn’t always guaranteed to work. 