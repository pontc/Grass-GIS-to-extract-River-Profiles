\chapter{Preparing data} \label{prep}

\section{Choosing your dataset}

Free geospatial data is widely available, so today's challenges is not finding it but deciding which dataset is best for the project at hand. There are a number of different topographic datasets freely available online, you will need to decide which is most appropriate for your project. LIDAR is preferable for detailed sudies with resolutions of approx 10m. Digital elevation models of 30 m (1 arc second) to 90 m (3 arc second) resolution are often used for regional to continental-scale drainage inversions. Computational power will be one consideration when deciding on the areal extent of your project. The following are high resolution datasets with global resolution:

\vspace{3mm}

\noindent	\textbf{Space Shuttle Radar Topography Mission (SRTM):} SRTM has global 30m and 90m datasets available on NASA's Earthdata portal:
\url{https://search.earthdata.nasa.gov/search}. 
NASA gathered the data using interferometric synthetic arpeture radar (inSAR) on an 11 day mission back in 2000 aboard its Space Shuttle Endeavour. Keep a look out for NASADEM which will be a fully reprocessed SRTM data using state-of-the-art interferometric processing techniques. The final version has not yet been released. 

\vspace{3mm}

\noindent \textbf{ASTER Global Digital Elevation Model:} ASTER GDEM-2 is also available for download at NASA Earthdata. ASTER GDEM is a collaboration between NASA and Japan's Advanced Spaceborne Thermal Emission and Reflection Radiometer (ASTER). The digital elevation model was created using stereoscopic pairs, using stereopairs and photogrammetry to measure elevation from two images at different angles. Cloud cover is the main problem with accuracy of the ASTER GDEM products but is considered to be a more accurate representation of rugged mountainous terrain than SRTM. 

\vspace{3mm}

\noindent \textbf{JAXA's Global ALOS 3D World:} Tiles can be downloaded from JAXA's portal on registration: \url{https://www.eorc.jaxa.jp/ALOS/en/aw3d30/}. This 30m digital surface model was collected aboard the Advanced Land Observing Satellite "ALOS" by the Panachromatic Remote-sensing Instrument for Stereo Mapping (PRISM), with the latest version released in 2019. 

\vspace{3mm}

\noindent \textbf{LIDAR:} The following website has a nice compilation of some of the global LIDAR data available to use: \url{https://arheologijaslovenija.blogspot.com/p/blog-page_81.html?spref=tw}. 


\section{Downloading STRM Granules}

Earthdata website provides a ready-made script to download your chosen granules directly onto your computer. Opt for direct download and click on ``Download Access Script``. This will take you to a page containing instructions on how to use the download script. Click the ``Download Script File`` button to download the script. 

\noindent If you are using Ubuntu, you will need to make the following change $\#$!/bin/sh $\rightarrow$ $\#$!/bin/bash as \textit{bin/sh} no longer links to bash and instead points to another shell called dash. Now execute and run the script from the command line in your Download folder. You will be prompted for your Earthdata username and password, be aware that nothing will show up when you type your password. The grids will now be automatically downloaded to the Downloads folder. Unzip files to you chosen destination folder.

\begin{lstlisting}[language=bash]
cd Downloads
chmod 777 download.sh 
./download.sh 
unzip "*.hgt" destination-folder
\end{lstlisting}


\section{Merging rasters}

By far, the simplest and most time efficient way is to use gdal${\_}$merge.py instead of ArcGIS's \textit{``Mosaic to New Raster"} tool to mosaic each of the SRTM granules into one grid. To give an example, 176 SRTM grids at 30m resolution covering the northern Arabian Peninsula was stitched together in less than a minute.

\begin{lstlisting}[language=bash]
ls *.hgt > DEMs.txt
gdal_merge.py -of Gtiff -n -32768 -a_nodata -32768 \
	-o merged_dem.tif --optfile DEMs.txt
\end{lstlisting}

\noindent This example uses an output format of Gtiff but can any format ca be specified using the \textit{-of} flag. No data values can be lost in translation so specify a no data value to mantain NoData cells in the merged dem. If you are using programs like RichDem to hydrologically process the DEM, this will be a pre-requisite. The \textit{-n} flag ignores any pixels in the files being merged with this pixel value while \textit{-a${\_}$nodata} assigns a specified nodata value to the output. You can check no data values in the original SRTM grids using \textit{gdalinfo}. This tool will list information about your raster dataset, including the raster’s coordinate system, resolution (i.e. pixel size) and regional extent. 

\vspace{3mm}

\noindent \textbf{Example}
\begin{lstlisting}
gdalinfo ~/openev/utm.tif
Driver: GTiff/GeoTIFF
Size is 512, 512
Coordinate System is:
PROJCS["NAD27 / UTM zone 11N",
    GEOGCS["NAD27",
        DATUM["North_American_Datum_1927",
            SPHEROID["Clarke 1866",6378206.4,294.978698213901]],
        PRIMEM["Greenwich",0],
        UNIT["degree",0.0174532925199433]],
    PROJECTION["Transverse_Mercator"],
    PARAMETER["latitude_of_origin",0],
    PARAMETER["central_meridian",-117],
    PARAMETER["scale_factor",0.9996],
    PARAMETER["false_easting",500000],
    PARAMETER["false_northing",0],
    UNIT["metre",1]]
Origin = (440720.000000,3751320.000000)
Pixel Size = (60.000000,-60.000000)
Corner Coordinates:
Upper Left  (  440720.000, 3751320.000) (117d38'28.21"W, 33d54'8.47"N)
Lower Left  (  440720.000, 3720600.000) (117d38'20.79"W, 33d37'31.04"N)
Upper Right (  471440.000, 3751320.000) (117d18'32.07"W, 33d54'13.08"N)
Lower Right (  471440.000, 3720600.000) (117d18'28.50"W, 33d37'35.61"N)
Center      (  456080.000, 3735960.000) (117d28'27.39"W, 33d45'52.46"N)
Band 1 Block=512x16 Type=Byte, ColorInterp=Gray
\end{lstlisting}


\section{Projecting data}

Before importing your raster layer into GRASS, use \textit{gdalwarp} to project the raster data into your preferred coordinate system. SRTM raster datasets uses a geographic coordinate system based on a spherical surface. This can be problematic when measuring distances in angular units as it is highly dependant on where you are on the Earth’s surface. Extracting rivers relies on accurately knowing river distances measured in length, so use an equal areas projected coordinate system. For small study areas, Universal Transvere Mercator (UTM) system is widely used while Albers Equal Areas or equivalent projections may be used for continent-wide analyses,. Note that you will also need to consider the appropriate ellipsoid. 

\begin{lstlisting}[language=bash]
proj='+proj=lcc +lat_1=17.0 +lat_2=33.0 +lat_0=25.08951 \
	+lon_0=48.0 +ellps=intl +units=m +no_defs'

gdalwarp -t_srs $proj merged_dem.tif projected_dem.tif
\end{lstlisting}

\noindent You can use -te \textless xmin ymin xmax ymax\textgreater  to clip the dem to a specific extent if you want. 

\section{Filling Voids}

The primary objective of the NASA MeaSUREs project (Making Earth System Data Records for Use in Research Environments) Program was to remove voids (no data holes) in the NASA SRTM DEM. In theory data should be seamless an no processing is required to fill voids. 

\noindent There is a tool in GRASS to check and fill no data voids in the dem. It is also possible to use the flag in gdal \textit{dstnodata -9999} to fix any issues although this isn’t always guaranteed to work. 