Intrduction

Landscapes
Input: tectonic plates or dynamic topography
[How to deconvolve the signal?]
Surface processes: Erosion and deposition

Main driver of erosion are river systems in unglaciated landscapes. 

Analysis of river long profiles key method to interpreting landscape evolution.

Concept of topographic steady state: Erosion balances stable rock uplift over long periods of time with steady climatic conditions. 

River systems tansmit signals upstream through the channel network while active faulting can drive additional signals through the system. These extrinsic signals are futher complicated by rheological heterogenity - fracture density or differential lithology. A rivers transient adjustment to these signals are as discrete changes in the river gradient referred to as a knickzone. 

Geomorphological intepretation of knickpoint formation is challenging. Origin of knickpoints have been attributed to tectonic events and faulting or climatically triggered base level fall. Smaller knickpoints can form as a result of different lithologies or sediment supply or hydraulic conditiosn. 

Interest:
Tectonic origin
LArge scale 'knickzones' of min 100m elevations
Over 0.5 Ma timescales

Break-in slope vs break-in-elecations 'stepped knickpoints. Only interested in break-in-slope. 